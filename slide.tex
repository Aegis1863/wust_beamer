% \documentclass[aspectratio=169]{beamer} % 16:9
\documentclass{beamer}
\usepackage{ctex, hyperref}
\usepackage[T1]{fontenc}
\graphicspath{{pic/}}

% other packages
\usepackage{latexsym, amsmath, xcolor, multicol, booktabs, calligra}
\usepackage{graphicx, pstricks, listings, stackengine}

\author[Sunbowen Lee]{李孙博闻}
\title[Graph neural network]{图神经网络的温和入门}
% \subtitle{图神经网络介绍}
\institute{武汉科技大学 \\ 理学院 \\ 冶金工业过程系统科学湖北省重点实验室}
\date{\today}
\usepackage{wust}

% 定义颜色
\def\cmd#1{\texttt{\color{red}\footnotesize $\backslash$#1}}
\def\env#1{\texttt{\color{blue}\footnotesize #1}}
\definecolor{deepblue}{rgb}{0,0,0.5}
\definecolor{deepred}{rgb}{0.6,0,0}
\definecolor{deepgreen}{rgb}{0.05, 0.44, 0.29}  % 青山绿
\definecolor{lightblue}{rgb}{0.63, 0.80, 0.86}  % 沁湖蓝

\lstset{
    basicstyle=\ttfamily\small,
    keywordstyle=\bfseries\color{deepblue},
    emphstyle=\ttfamily\color{deepred},    % Custom highlighting style
    stringstyle=\color{deepgreen},
    numbers=left,
    numberstyle=\small\color{halfgray},
    rulesepcolor=\color{red!20!green!20!blue!20},
    frame=shadowbox,
}

\begin{document}

\kaishu
\begin{frame}  % 封面
    \titlepage
    \begin{figure}[htpb]
        \centering
        \vspace{-0.7cm}
        \includegraphics[width=0.45\linewidth]{wust.png}
    \end{figure}
\end{frame}

\begin{frame}{前言}
    尽管图网络具体实现可通过简单调包实现,但在此之前仍然需要了解图的基础知识,特别是``图嵌入''和``信息传递''的概念。\newline

    本幻灯片材料主要参考Sanchez-Lengeling\cite{sanchez-lengeling2021a}等人的论文.
\end{frame}

\begin{frame}{目录}
    \frametitle{Table of Contents}
    \tableofcontents
\end{frame}

\AtBeginSubsection[]{
	\begin{frame}{目录}
		\frametitle{Table of Contents}
		\tableofcontents[currentsubsection]
	\end{frame}
}

\section{图结构基础}

% \begin{itemize} % [<+-| alert@+>]

\begin{frame}{Vertex, Edge and Global}
    \begin{figure}
        \includegraphics[width=\textwidth]{vertex.png}
    \end{figure}
\end{frame}

\begin{frame}{Vertex, Edge and Global}
    \begin{figure}
        \includegraphics[width=\textwidth]{edge.png}
    \end{figure}
\end{frame}

\begin{frame}{Vertex, Edge and Global}
    \begin{figure}
        \includegraphics[width=\textwidth]{global.png}
    \end{figure}
    注意全局用U而非G表示。
\end{frame}

\begin{frame}{Vertex, Edge and Global}
    \begin{figure}
        \includegraphics[width=\textwidth]{graph.png}
    \end{figure}
    图的三种元素都可以包含嵌入向量,这里嵌入是一维的,可视化呈现。
\end{frame}

\section{哪里有图结构?}

\begin{frame}{图像}
    \begin{figure}
        \includegraphics[width=0.5\textwidth]{图片像素.jpg}
        \caption{图片是由许多像素点构成的,每个像素点有一个或多个数值,如灰度值或者RGB值。}
    \end{figure}
\end{frame}

\begin{frame}{从图像到图}
    \begin{figure}
        \includegraphics[width=\textwidth]{图1.jpg}
        \caption{左边是图像,中间是邻接矩阵,右边是图。}
    \end{figure}
    图像也可以表示为图,一个5*5的图像可以表示为邻接矩阵和图的形式。这里三种不同的表现方法是同一个信息的不同表示方法。
\end{frame}

\begin{frame}{句子表示为图}
    \begin{figure}
        \includegraphics[width=\textwidth]{图2.jpg}
        \caption{语句表现为图的形式。}
    \end{figure}
    这里邻接矩阵去掉了对角连接和下三角表示。另外,语句在机器学习中通常不会用图表示,而是通过编码并映射到嵌入向量。
\end{frame}

\begin{frame}{分子结构图}
    \begin{figure}
        \includegraphics[width=\textwidth]{图3.jpg}
        \caption{分子结构表现为图。}
    \end{figure}
    相比于前面的图,分子图更具有异质性。
\end{frame}

\begin{frame}{人物关系图}
    \begin{figure}
        \includegraphics[width=\textwidth]{图4.jpg}
        \caption{人物关系建模为图。}
    \end{figure}
    话剧奥赛罗中的人物关系可以建模为图,节点表现为角色,边建模为人物间联系。
\end{frame}

\begin{frame}{论文引用关系图}
    \begin{figure}
        \includegraphics[width=0.5\textwidth]{图5.jpg}
        \caption{文章引用表示为图}
    \end{figure}
\end{frame}

\section{图任务}

\begin{frame}{图分类任务}
    \begin{figure}
        \includegraphics[width=\textwidth]{全局分类.jpg}
        \caption{识别哪些分子有两个苯环}
    \end{figure}
    输入多个分子的图信息,输出各个图的类别。
\end{frame}

\begin{frame}{点分类任务}
    \begin{figure}
        \includegraphics[width=\textwidth]{点分类.jpg}
        \caption{识别那些人支持John H,哪些人支持Hi。}
    \end{figure}
    输入一个图,输出各个点的类别,这里是二分类。
\end{frame}

\begin{frame}{边分类任务}
    \begin{figure}
        \includegraphics[width=0.7\textwidth]{边分类.jpg}
        \caption{先对图片做分割,然后把被分割的部分作为节点,识别各个节点之间的关系。对局者站在垫子上,对局者相互对抗,裁判和观众观看对抗。}
    \end{figure}
\end{frame}

\begin{frame}{边分类任务}
    \begin{figure}
        \includegraphics[width=\textwidth]{边分类2.jpg}
        \caption{图 9 简化图。}
    \end{figure}
\end{frame}

\section{图的构建}

\begin{frame}{邻接矩阵不足以表示图信息}
    \begin{figure}
        \includegraphics[width=0.9\textwidth]{邻接矩阵.jpg}
        \caption{邻接矩阵经过初等变换,即行变换或者列变换,其含义并没有改变,但是矩阵特征发生很大改变。}
    \end{figure}
\end{frame}

\begin{frame}{邻接矩阵不足以表示图信息}
    \begin{figure}
        \includegraphics[width=0.9\textwidth]{图7.jpg}
        \caption{简单例子}
    \end{figure}
\end{frame}

\begin{frame}{图嵌入}
    下面的图片展示了图如何构建为嵌入向量, 需要注意的是, 在邻接表(edge\_index)中, 哪两个节点有连接是按列表示的, 也同时按照顺序定义了方向.
    \begin{figure}
        \includegraphics[width=\textwidth]{graph_sample.png}
        \caption{邻接矩阵无法用于学习,但是图可以转化为嵌入向量形式以学习。通过edge\_index、node\_feature等若干嵌入向量可以准确表示图信息。}\label{图嵌入}
    \end{figure}
\end{frame}

\begin{frame}{图嵌入}
    \begin{itemize}[<+-| alert@+>]
        \item 以节点举例,图 \ref{图嵌入} 中一个节点的属性可以表示为-1,0,-1三个数字,也可以表示为任意维度的张量,其他元素同理。
        \item 如果一个节点表示一个学生,它的节点属性可以是包含多个信息的向量,比如[性别、班级、学校]。
        \item 如果要对节点分类,可以给出节点标签;如果是对图分类,可以给出图标签,此时节点标签就不是必要的,比较重要的是节点特征和邻接表:节点特征是图的基本信息,邻接表引定义图结构。
        \item 这里的邻接表可以表示为其他不同的形式,以使用的图网络的包的要求为准。
    \end{itemize}
\end{frame}

\section{图神经网络}

\subsection{处理图结构的基础方法}

\begin{frame}{分开处理每个嵌入向量}
\begin{figure}
    \centering
    \includegraphics[width=0.9\textwidth]{简易GNN.png}
    \caption{简易GNN。前面我们已经把图转化成包含点信息、边信息、全局信息的嵌入形式,因此我们可以分别构建三个神经网络去学习它们各自的特征。另外,这里图神经网络没有改变节点直接的连接性,因此经过处理前后的邻接矩阵也不会变化。}
\end{figure}
\end{frame}

\begin{frame}{基础点分类}
    \begin{figure}
    \centering
    \includegraphics[width=0.6\textwidth]{点分类例子.png}
    \caption{基础点分类图解。把点特征整理为一个矩阵,点标签整理为列向量,发现与普通机器学习并无区别。边分类和图分类同理。}
\end{figure}
\end{frame}

\subsection{基础图神经网络}

\begin{frame}{信息传递}
    \textbf{信息传递是图神经网络的关键要素.}
    \begin{figure}
        \centering
        \includegraphics[width=0.5\textwidth]{信息传递.png}
        \caption{信息传递机制. 信息传递也叫聚合. 一个节点可以从与其连接的边或者节点聚合信息. 边也可以聚合节点的信息. 边嵌入和点嵌入至少要有一个. 聚合操作一般是指池化. 聚合之后同样可以通过MLP学习.}
    \end{figure}
    问题:1. 池化会不会损失太多信息?
\end{frame}

\begin{frame}{节点信息聚合到全局}
    \begin{figure}
        \includegraphics[width=\textwidth]{节点预测全局.png}
        \caption{节点预测全局. 对全局节点做池化聚合到全局嵌入, 可以实现全局信息聚合. 同理还可以聚合边的信息到全局.}
    \end{figure}
\end{frame}

\begin{frame}{比较复杂的聚合}
    \begin{figure}
        \includegraphics[width=0.9\textwidth]{节点到边到节点聚合.png}
        \caption{节点-边-节点聚合.}
    \end{figure}
\end{frame}

\begin{frame}{比较复杂的聚合}
    \begin{figure}
        \includegraphics[width=\textwidth]{节点边全局聚合.png}
        \caption{节点-边-全局混合聚合.}
    \end{figure}
\end{frame}

\begin{frame}
    \begin{itemize}
        \item 池化聚合是最简单的方法之一, 图卷积方法是现在的主流聚合方案, 在此基础上很容易加入注意力机制. 
        \item 值得注意的是图卷积和图像卷积在操作上几乎没有相同点.
        \item 创新聚合方法是一种主要的改进方向.
    \end{itemize}
\end{frame}

\begin{frame}
    \begin{figure}
        \includegraphics[width=0.5\textwidth]{有向图.png}
        \caption{一种有向图. 这表示边是单向的, 消息传递只能按照边方向进行. 其中红色节点包括1-hop邻居和2-hop邻居.}
    \end{figure}
    \begin{block}{Note}
        无向图是指边双向连接而不是无方向连接.
    \end{block}
\end{frame}

\begin{frame}
    \begin{figure}
        \includegraphics[width=0.9\textwidth]{两步消息传递.png}
        \caption{对于红色节点, 通过两次信息传递就可以获得全局信息.}
    \end{figure}
    每一次信息传递都伴随一次神经网络处理, 因此这里的step可以看作神经网络中的epoch, 两步消息传递也就是两个epoch. 这里仅构建了有向图, 对于无向图来说, 每个节点每一步都在聚合邻居信息.
\end{frame}

\section{总结}

\begin{frame}
    \begin{itemize}
        \item 图需要转化为嵌入向量才能被处理;
        \item 图非常擅长描述具有拓扑结构的数据;
        \item 图神经网络的特点在于信息传递;
    \end{itemize}
\end{frame}

\section{参考文献}

\begin{frame} %[allowframebreaks]
    \bibliography{ref}
    \bibliographystyle{plain}
    % 如果参考文献太多的话,可以像下面这样调整字体:
    % \tiny\bibliographystyle{alpha}
\end{frame}

\begin{frame}
    \begin{center}
        {\Huge\calligra Thanks!}
    \end{center}
\end{frame}

\end{document}